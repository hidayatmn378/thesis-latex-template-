\documentclass[12pt, a4paper, onecolumn, oneside, final]{report}
\usepackage[latin1]{inputenc}
%\usepackage[indonesian]{babel}
\usepackage{amsmath}
\usepackage{amsfonts}
\usepackage{amssymb}
\usepackage{tocloft}
\usepackage[left= 3.5cm,right=3cm,top=3cm,bottom=3cm]{geometry}
\usepackage{indentfirst}
\usepackage{amsmath,amssymb,amsfonts,amsthm}
\usepackage{array}
\usepackage{caption}
\usepackage{wrapfig}
\usepackage{graphicx}
\usepackage{varwidth}
\usepackage{float}
\usepackage{indentfirst}
\usepackage{textcomp}
\usepackage{lmodern}
\usepackage{enumerate}
\usepackage{tabularx}
\usepackage{microtype}
%\usepackage[framed]{matlab-prettifier}
\usepackage{inputenc}
\usepackage{tikz}
\usepackage{xcolor,colortbl}
\usepackage{multirow}
\usepackage{times}
\usepackage{float}
\usepackage{hyperref}
\usepackage{setspace}
%\usepackage{apacite}
\usepackage{enumitem}
\usepackage[labelsep=period]{caption}
%\usepackage{pbsi}
\usepackage[T1]{fontenc}
\usepackage{esint}
\usepackage{lipsum}
%\usepackage[justification=centering]{caption}

%%
% Hyphenation untuk Indonesia
%
% @author  Andreas Febrian
% @version 2.02
% @edit by Ichlasul Affan
%
% Tambahkan cara pemenggalan kata-kata yang salah dipenggal secara otomatis
% oleh LaTeX. Jika kata tersebut dapat dipenggal dengan benar, maka tidak
% perlu ditambahkan dalam berkas ini. Tanda pemenggalan kata menggunakan
% tanda '-'; contoh:
% menarik
%   --> pemenggalan: me-na-rik
%


% Silakan ganti ke bahasa Inggris (\selectlanguage{english}) jika Anda merasa terlalu banyak kata bahasa Inggris yang pemenggalannya tidak benar.
\selectlanguage{indonesian}


\hyphenation{
    % alphabhet A
    a-na-li-sa a-tur
    a-pli-ka-si
    % alphabhet B
    ba-ngun-an
    be-be-ra-pa
    ber-ge-rak
    ber-ke-lan-jut-an
    ber-pe-nga-ruh
    % alphabhet C
    ca-ri
    % alphabhet D
    di-da-pat-kan di-sim-pan di-pim-pin de-ngan da-e-rah di-ba-ngun da-pat di-nya-ta-kan
    di-sim-bol-kan di-pi-lih di-li-hat de-fi-ni-si di-de-fi-ni-si-kan di-mi-li-ki
    % alphabhet E
    e-ner-gi eks-klu-sif
    % alphabhet F
    fa-si-li-tas
    % alphabhet G
    ga-bung-an ge-rak
    % alphabhet H
    ha-lang-an
    % alphabhet I
    % alphabhet J
    % alphabhet K
    ke-hi-lang-an
    ku-ning
    kua-li-tas ka-me-ra ke-mung-kin-an ke-se-pa-ham-an
    % alphabhet L
    ling-kung-an
    % alphabhet M
    me-neng-ah
    meng-a-tas-i me-mung-kin-kan me-nge-na-i me-ngi-rim-kan
    meng-u-bah meng-a-dap-ta-si me-nya-ta-kan mo-di-fi-ka-si
    meng-a-tur meng-a-rah-kan mi-lik
    % alphabhet N
    nya-ta non-eks-klu-sif
    % alphabhet O
    % alphabhet P
	pe-nye-rap-an
	pe-ngon-trol
    pe-mo-del-an
    pe-ran  pe-ran-an-nya
    pem-ba-ngun-an pre-si-den pe-me-rin-tah prio-ri-tas peng-am-bil-an
    peng-ga-bung-an pe-nga-was-an pe-ngem-bang-an
    pe-nga-ruh pa-ra-lel-is-me per-hi-tung-an per-ma-sa-lah-an
    pen-ca-ri-an pen-ce-ta-kan peng-struk-tur-an pen-ting pen-ting-nya
    % alphabhet Q
    % alphabhet R
    ran-cang-an
    % alphabhet S
    si-mu-la-si sa-ngat
    % alphabhet T
    te-ngah
    ter-da-pat
    trans-for-ma-si
    % alphabhet U
    % alphabhet V
    va-ri-an va-ri-a-si
    % alphabhet W
    % alphabhet X
    % alphabhet Y
    % alphabhet Z
    % special
}

\usepackage{amsmath,amssymb,amsfonts,amsthm}
\renewcommand{\chaptername}{BAB}
%\usepackage{setspace}

\setcounter{tocdepth}{1}

\definecolor{green}{rgb}{0.1,0.1,0.1}
\newcommand{\done}{\cellcolor{teal}done}  %{0.9}
\newcommand{\hcyan}[1]{{\color{teal} #1}}

\newtheorem{theorem}{Dalil}
\newtheorem{corollary}{Akibat}
\newtheorem{lemma}{Lemma}
\newcommand\at[2]{\left.#1\right|_{#2}}
\theoremstyle{definition}
\newtheorem{definition}{Definisi}{}
\newtheorem{example}{\textsf{Contoh}}

\newenvironment{bukti}[1][Bukti]{\noindent{\it\textit{#1. }}}
{\hspace{\stretch{1}}\rule{.5em}{.5em}}

\DeclareMathOperator{\mo}{mod\,}
\DeclareMathOperator{\ord}{ord}
\DeclareMathOperator{\fpb}{fpb}
\newcommand{\defi}{\overset{\mbox{\tiny{\sf def}}}{=}}
\newcommand{\znz}{\Bbb Z/n\Bbb Z}\newcommand{\zpz}{\Bbb Z/p\Bbb Z}
\numberwithin{equation}{chapter}

\setcounter{tocdepth}{3}
\setcounter{secnumdepth}{3}

\setlength\parindent{12.5mm}

\usepackage{blindtext}
\renewcommand\cftbeforetoctitleskip{-1cm}
 \renewcommand\cftbeforeloftitleskip{-1cm}
 \renewcommand\cftbeforelottitleskip{-1cm}
\renewcommand{\cftdotsep}{1}
\renewcommand{\cftchapleader}{\cftdotfill{\cftsecdotsep}}

\renewcommand{\contentsname}{DAFTAR ISI}
\renewcommand{\cfttoctitlefont}{\hfil\Large\bfseries\MakeUppercase}
\renewcommand{\cftchapfont}{\bfseries}
\renewcommand{\cftchappagefont}{\bfseries}
\renewcommand{\cftchappresnum}{BAB }
\renewcommand{\cftchapnumwidth}{3.7em}

\renewcommand{\cftlottitlefont}{\hfil\large\bfseries\MakeUppercase}
\renewcommand{\cfttabpresnum}{Tabel }
\renewcommand{\cfttabnumwidth}{6em}

\renewcommand{\cftloftitlefont}{\hfil\large\bfseries\MakeUppercase}
\renewcommand{\cftfigpresnum}{Gambar }
\renewcommand{\cftfignumwidth}{6em}

\makeatletter
\def\ps@myPS{%
    \def\@oddfoot{\null\hfill\thepage}
    \def\@evenfoot{\thepage}%
    \def\@evenhead{\null\hfil\slshape\leftmark}%
    \def\@oddhead{{\slshape\rightmark}}}%
\makeatother

\makeatletter % default is "\newcommand\@chapapp{\chaptername}"
\renewcommand\@chapapp{\textls[40]{\MakeUppercase{\chaptername}}}
\makeatletter

\usepackage{titlesec}
% 1. Judul bab ditengah
\titleformat{\chapter}[display]
 {\normalfont\large\bfseries\centering}
 {\chaptertitlename\ \Roman{chapter}}{0pt}{\large}
% 2. Font section 12pt dan tambah titik section, so 1.1 menjadi 1.1.
\titlespacing{\chapter}{0pt}{50pt}{\baselineskip}
\titleformat{\section}
 {\normalfont\fontsize{12}{15}\bfseries}{\thesection.}{1em}{}
% 3. Font subsection 12pt dan tambah titik subsection
 \titleformat{\subsection}
 {\normalfont\fontsize{12}{15}\bfseries}{\thesubsection.}{1em}{} 
% 4. Hapus spasi setelah bab
\makeatletter
\def\ttl@mkchap@i#1#2#3#4#5#6#7{%
 \ttl@assign\@tempskipa#3\relax\beforetitleunit
 \vspace{\@tempskipa}%<<<<<< REMOVE THE * AFTER \vspace
 \global\@afterindenttrue
 \ifcase#5 \global\@afterindentfalse\fi
 \ttl@assign\@tempskipb#4\relax\aftertitleunit
 \ttl@topmode{\@tempskipb}{%
 \ttl@select{#6}{#1}{#2}{#7}}%
 \ttl@finmarks % Outside the box!
 \@ifundefined{ttlp@#6}{}{\ttlp@write{#6}}}
 
 \usetikzlibrary{shapes.geometric, arrows}

\newcommand{\listappendicesname}{DAFTAR LAMPIRAN}
\newlistof{appendices}{apc}{\listappendicesname}
\newcommand{\appendices}[1]{\addcontentsline{apc}{appendices}{#1}}
\renewcommand\cftbeforeapctitleskip{-1cm}
\renewcommand{\cftapctitlefont}{\hfil\large\bfseries\MakeUppercase}

\newcommand{\newappendix}[1]{\section*{#1}\appendices{#1}}

% Tambah kata ejaan yang salah di Latex 
\hyphenation{
    alphabhet A 
    alphabhet B 
    alphabhet C 
    alphabhet D 
    alphabhet E
    alphabhet F 
    alphabhet G 
    alphabhet H 
    alphabhet I
    alphabhet J
    alphabhet K 
    alphabhet L
    alphabhet M 
    alphabhet N 
    alphabhet O
    alphabhet P 
    alphabhet Q
    alphabhet R
    alphabhet S 
    alphabhet T 
    alphabhet U
    alphabhet V
    alphabhet W 
    alphabhet X
    alphabhet Y
    alphabhet Z
    special}
  \makeatletter
\renewcommand*\@pnumwidth{3em}
\makeatother

%=====================================================================
\begin{document}

\tikzstyle{rect} = [draw, rectangle, fill=white!20, text width=25em, text centered, minimum height=2em]%text width=20em/13em
\tikzstyle{elli} = [draw, ellipse, fill=white!20, minimum height=2em]
\tikzstyle{circ} = [draw, circle, fill=white!20, minimum height=2em, inner sep=10pt]
\tikzstyle{diam} = [draw, diamond, fill=white!20, text width=6em, text badly centered, inner sep=0pt]
\tikzstyle{line} = [draw, -latex']
%=====================================================================
\pagenumbering{roman}
%=====================================================================
% Halaman Judul
\addtocontents{toc}{~\hfill{\it Halaman}\par}
\addtocontents{toc}{\begingroup\protect\setlength{\protect\cftsecindent}{-\leftmargin}}
\addcontentsline{toc}{section}{\protect\numberline{}Judul}
\addtocontents{toc}{\endgroup}
\begin{spacing}{1}
\begin{center}
{\Large\textbf{TITLE}}\\[1.0cm]
\end{center}
\vspace*{0.8cm} 

\begin{center}

% harus dalam 16pt Times New Roman
\large{\textbf{TESIS/DISERTASI}}
\\\vspace*{1.8cm}    
\normalsize{Sebagai Salah Satu Syarat Untuk Memperoleh \\
Gelar * Ilmu * \\
pada Program *}\\[1.5cm]
%\vspace*{0.3 cm}    
%% harus dalam 16pt Times New Roman
\vspace*{1cm}  
{\large Oleh:}\\
\vspace*{1cm}       
% penulis dan NIM
\large{\textbf{\underline{NAMA}}}
\\\large{\textbf{NPM}} 
\end{center}\vspace*{1cm}   

\begin{figure}[h]
\centering
\includegraphics[width=5cm]{USK} %logo universitas
\end{figure}
\vspace*{1.5cm}   

\begin{center}
% informasi mengenai fakultas dan program studi
\textbf{PROGRAM *\\
PROGRAM PASCASARJANA UNIVERSITAS *\\
Kota - Kota\\
2021}
\end{center}
\thispagestyle{empty}
	\end{spacing}
%=====================================================================
% Halaman Pengesahan
\pagebreak
\chapter*{PENGESAHAN}
\addtocontents{toc}{\begingroup\protect\setlength{\protect\cftsecindent}{-\leftmargin}}
\addcontentsline{toc}{section}{\protect\numberline{}Pengesahan}	
\addtocontents{toc}{\endgroup}
\setcounter{page}{2}
%\begin{spacing}{1.5}
\begin{center}
{\large\textbf{TITLE}}\\[1cm]
\end{center}

\begin{center}
{\large\textbf{\textit{TITLE}}}\\[1cm]
\end{center}  
\begin{center}
{\large Oleh}
\end{center}

\begin{center}
\normalsize
\noindent
\begin{tabular}{l l l}
    Nama \verb"  " &: * \\
	NPM &: * \\
	Program Studi	&: * \\ 
\end{tabular} \\
\end{center}
%    \end{spacing}
\begin{center}
\vspace{0.5cm}
Menyetujui:\\%[0.5cm]

\vspace{1cm}

\begin{tabular}{l l }
Pembimbing I,\verb"             " & Pembimbing II, \verb"            "\\[2.25cm]
\underline{Nama1} & \underline{Nama2}\\
NIP. * & NIP. *
\end{tabular}
\end{center}

\begin{center}
\vspace{0.5cm}
Mengetahui:\\%[0.5cm]

\vspace{1cm}

\begin{tabular}{l l }
Dekan Fakultas *\verb"             " & \verb" "Ketua Prodi *\\
Universitas *, & \verb" "Universitas *,\\[2.25cm]
\underline{Nama1} & \verb" "\underline{Nama2}\\
NIP. * & \verb" "NIP. *
\end{tabular}
\end{center}
\vspace{0.3cm}
\begin{center}
*
\end{center}
\thispagestyle{empty}

%\input{LPS}
%=====================================================================
% Halaman Bebas Plagiasi
\pagebreak
\chapter*{PERNYATAAN BEBAS PLAGIASI}
\addtocontents{toc}{\begingroup\protect\setlength{\protect\cftsecindent}{-\leftmargin}}
\addcontentsline{toc}{section}{\protect\numberline{}Pernyataan Bebas Plagiasi}	
\addtocontents{toc}{\endgroup}
\begin{spacing}{1.5}
	\pagestyle{empty}
	\begin{center}
		\vskip 1cm
		\lipsum[1-3]
	\end{center}
\end{spacing}
\pagestyle{empty}
%=====================================================================
% Halaman Abstrak
\pagebreak
\chapter*{ABSTRAK}
\addtocontents{toc}{\begingroup\protect\setlength{\protect\cftsecindent}{-\leftmargin}}
\addcontentsline{toc}{section}{\protect\numberline{}Abstrak}	
\addtocontents{toc}{\endgroup}
\begin{spacing}{1.5}
	\pagestyle{empty}
	\begin{center}
		\vskip 1cm
		\lipsum[1-2]
	\end{center}
\end{spacing}
\pagestyle{empty}
%=====================================================================
% Halaman Kata Pengantar
\pagebreak
\chapter*{KATA PENGANTAR}	
\addtocontents{toc}{\begingroup\protect\setlength{\protect\cftsecindent}{-\leftmargin}}
\addcontentsline{toc}{section}{\protect\numberline{}Kata Pengantar}
\addtocontents{toc}{\endgroup}
\begin{spacing}{1.5}
	\pagestyle{empty}
	\begin{center}
		\vskip 1cm
		\lipsum[1-3]
	\end{center}
\end{spacing}
\pagestyle{empty}
%=====================================================================
% Halaman Daftar Isi
\pagebreak
\addtocontents{toc}{\begingroup\protect\setlength{\protect\cftsecindent}{-\leftmargin}}
\addcontentsline{toc}{section}{\protect\numberline{}Daftar Isi}	
\addtocontents{toc}{\endgroup}
\tableofcontents
%=====================================================================
% Halaman Daftar Tabel
\pagebreak
\addtocontents{toc}{\begingroup\protect\setlength{\protect\cftsecindent}{-\leftmargin}}
\addcontentsline{toc}{section}{\protect\numberline{}Daftar Tabel}	
\addtocontents{toc}{\endgroup}
\addtocontents{lot}{~\hfill{\it Halaman}\par}
\listoftables
%=====================================================================
% Halaman Daftar Gambar
\pagebreak
\addtocontents{lof}{~\hfill{\it Halaman}\par}
\addtocontents{toc}{\begingroup\protect\setlength{\protect\cftsecindent}{-\leftmargin}}
\addcontentsline{toc}{section}{\protect\numberline{}Daftar Gambar}	
\addtocontents{toc}{\endgroup}
\listoffigures
%=====================================================================
% Halaman Daftar Simbol
\pagebreak
\chapter*{DAFTAR SIMBOL}
\addtocontents{toc}{\begingroup\protect\setlength{\protect\cftsecindent}{-\leftmargin}}
\addcontentsline{toc}{section}{\protect\numberline{}Daftar Simbol}	
\addtocontents{toc}{\endgroup}
\vspace{1.5pc}
\vspace{1.5pc}
\begin{center}
\begin{tabular}{lp{0.75\textwidth}}
  $A$ & Amplitudo \\
  $A,A_x,A_y,A_z$ & Vektor A dan vektor satuannya \\
  $a,b,c,d$ & Konstanta \\
  $A_{eff}$ & Mode area efektif \\
  $\alpha$ & Koefisien atenuasi pada fiber optik\\
  $\mathbf{B}$ & Medan magnet \\
  $B,B_x$ & Vektor B dan vektor satuannya\\
  $\beta$ & Parameter dispersi gelombang grup pada persamaan nonlinier Schr\"odinger \\
  $\beta_2$ & dispersi gelombang grup pada fiber optik\\
  $B_j$ & Kekuatan osilator\\
  $c$ & Kecepatan cahaya\\
    $c.c$ & komplex sekawan (\textit{complex conjugate})\\
  $C,C_x,C_y$ & Vektor C dan vektor satuannya\\
  $\mathbf{D}$ & Pergeseran medan listrik\\
  $\Delta$ & Operator yang menyatakan perubahan\\
  $\partial$ & Turunan parsial\\
  $\mathbf{E}$ & Medan listrik\\
  $\hat{E}$ & Invers transformasi Fourier pada medan listrik\\
  $\langle E\rangle$ & Operator energi\\
  $\epsilon_0$ & Konstanta permitivitas\\
  $\epsilon_{NL}$ & Konstanta dielektrik nonlinier\\
  $F(t)$ & Transformasi Fourier\\
  $\hat{F}(\omega)$ & Invers transformasi Fourier\\
  $H$ & Operator Hamiltonian\\
    $\mathbf{H}$ & Intensitas medan magnet\\
      $h$ & Konstanta Planck\\
  $I$ & Matriks identitas\\
    $\bar{I}(t)$ & Intensitas gelombang elektromagnetik yang berubah terhadap waktu\\
  $Im$ & Suku imajiner \\
  $i$ & Bilangan imajiner \\
  $\mathcal{J}$ & Matriks Jacobian \\
  $\mathbf{J}$ & Rapat arus pada persamaan Maxwell \\
  $\tilde{K}$ & Operator energi kinetik \\
  $k$ & Konstanta propagasi pada ansatz persamaan nonlinier Schr\"odinger \\
  $k_0$ & Konstanta propagasi pada gelombang elektromagnetik \\
  $L$ & Panjang kabel fiber optik\\
  $l$ & Jarak tempuh persatuan waktu\\
    $ln$ & Logaritma natural \\
  $\lambda$ & Panjang gelombang\\

\end{tabular}
\end{center}

\begin{center}
\begin{tabular}{lp{0.75\textwidth}}
  $\lambda\tilde{X}$ & Nilai eigen yang bekerja pada vektor eigen\\
  $M$ & Polarisasi induksi medan magnet\\
  $\mu_0$ & Konstanta permeabilitas\\
  $m$ & Massa klasik soliton\\
  $n$ & Indeks bias refraktif\\
  $\bar{n}$ & Indeks bias inti fiber optik\\
  $\tilde{n}$ & Efek indeks refraktif\\
  $\nabla$ & Operator diferensial berbasis vektor\\
  $n_g$ & Indeks bias group\\
  $\omega$ & Frekuensi sudut tak berdimensi\\
  $\omega'$ & Frekuensi sudut tak berdimensi \\
  $\omega_j$ & Frekuensi resonansi\\
  $P$ & Polarisasi induksi medan listrik\\
  $\mathbf{P}$ & Daya\\
  $\hat{P}$ & Inversi transformasi Fourier untuk polarisasi\\
  $p$ & Momentum\\
  $\mathbf{P}_{in}$ & Besar daya yang masuk\\
  $\mathbf{P}_{out}$ & Besar daya yang keluar\\

    $P_L$ & Polarisasi linier\\
  $P_{NL}$ & Polarisasi nonlinier\\
  $\Psi$ & Fungsi gelombang pada persamaan Schr\"odinger\\
  $\psi(t)$ & Amplitudo persamaan nonlinier Schr\"odinger yang mengandung variabel waktu\\
  $\phi(t)$ & Transformasi persamaan nonlinier Schr\"odinger ke dalam bentuk persamaan diferensial biasa orde 1 (pemisalan dalam perumusan sistem persamaan diferensial)\\
  $\rho$ & Massa jenis (Densitas) medium\\
  $\sigma$ & Parameter suseptibilitas bahan pada persamaan nonlinier Schr\"odinger \\
  $T$ & Waktu tunda (\textit{delay time}) \\
  $t$ & Variabel waktu tak berdimensi \\
  $t'$ & Variabel waktu tak berdimensi \\
  $\langle U\rangle$ & Kerapatan energi \\
  $\tilde{V}$ & Operator potensial \\
  $V$ & Volume medium \\
  $\nu_g$ & Gelombang grup \\
  $x$ & Variabel spasial berdimensi-$x$\\
  $\dot{x}$ & Turunan pertama fungsi $x$ \\
  $\chi$ & Parameter suseptibilitas bahan\\
  $\chi^{(1)}$ & Parameter suseptibilitas bahan orde-1\\
  $\chi^{(2)}$ & Parameter suseptibilitas bahan orde-2\\
  $\chi^{(3)}$ & Parameter suseptibilitas bahan orde-3\\
  $\chi^{(n)}$ & Parameter suseptibilitas bahan orde-n\\
  $y$ & Variabel spasial berdimensi-$y$\\
  $\dot{y}$ & Turunan pertama fungsi $y$\\
  \end{tabular}
\end{center}

\begin{center}
\begin{tabular}{lp{0.75\textwidth}}
  $\hat{y}$ & vektor polarisasi\\
  $z$ & Variabel spasial tak berdimensi\\
\end{tabular}
\end{center}
%=====================================================================
% Halaman Daftar Lampiran
\pagebreak
\listofappendices
\addtocontents{apc}{~\hfill{\it Halaman}\par}
\addtocontents{toc}{\begingroup\protect\setlength{\protect\cftsecindent}{-\leftmargin}}
\addcontentsline{toc}{section}{\protect\numberline{}Daftar Lampiran}	
\addtocontents{toc}{\endgroup}
%=====================================================================
% Halaman Persembahan
\pagebreak
\begin{spacing}{1.5}
	\pagestyle{empty}
	\begin{center}
		\vskip 1cm
		\lipsum[1-3]
	\end{center}
\end{spacing}
\pagestyle{empty}
%=====================================================================
\newpage
\makeatother
\newpagestyle{chapterpage}{\setfoot{}{}{\thepage}}
\assignpagestyle\chapter{chapterpage}
\setcounter{page}{1}
\pagenumbering{arabic}
\pagestyle{myPS}
\def\thechapter{\Roman{chapter}} 
\def\thesection{\arabic{chapter}.\arabic{section}}
\def\thesubsection{\arabic{chapter}.\arabic{section}.\arabic{subsection}}
\def\theequation{\arabic{chapter}.\arabic{equation}}
\def\thefigure{\arabic{chapter}.\arabic{figure}}
\def\thetable{\arabic{chapter}.\arabic{table}}
%=====================================================================% BAB I
\chapter{PENDAHULUAN}
\vspace{1.5pc}
\vspace{1.5pc}
\section[Latar Belakang]{LATAR BELAKANG}
\begin{spacing}{1.5}
	
\lipsum[1-4]
	
\section[Rumusan Masalah]{RUMUSAN MASALAH}

\lipsum[1-2]

\section[Tujuan Penelitian]{TUJUAN PENELITIAN}

\lipsum[1-2]

\section[Manfaat Penelitian]{MANFAAT PENELITIAN}

\lipsum[1-2]

\end{spacing}
%=====================================================================
% BAB II
\chapter{TINJAUAN KEPUSTAKAAN}
\vspace{1.5pc}
\vspace{1.5pc}
\section[State of Art]{STATE OF ART}
\begin{spacing}{1.5}
	
\lipsum[1-4]
	
\end{spacing}
%=====================================================================
% BAB III
\chapter{METODOLOGI PENELITIAN}
\vspace{1.5pc}
\vspace{1.5pc}
\section[Metode Penelitian]{METODE PENELITIAN}
\begin{spacing}{1.5}
	
	\lipsum[1-4]
	
\end{spacing}
%=====================================================================
% BAB IV
\chapter{HASIL DAN PEMBAHASAN}
\vspace{1.5pc}
\vspace{1.5pc}
\section[Hasil 1]{HASIL 1}
\begin{spacing}{1.5}
	
	\lipsum[1-4]
	
\end{spacing}
%=====================================================================
% BAB V
\chapter{PENUTUP}
\vspace{1.5pc}
\vspace{1.5pc}
\begin{spacing}{1.5}
\section[Kesimpulan]{KESIMPULAN}

\lipsum[2-4]

\section[Saran]{SARAN}

\lipsum[2-4]

\end{spacing}
%=====================================================================
% Halaman Daftar Pustaka
\pagebreak
\addcontentsline{toc}{chapter}{\textbf{DAFTAR PUSTAKA}}
%\input{DAFTAR}
%=====================================================================
% Halaman Lampiran
\newpage
\newappendix{Lampiran 1. \textit{Listing Program}}
\addcontentsline{toc}{chapter}{LAMPIRAN}
%\lstinputlisting[style = Matlab-editor]{kombinasi.m}
\pagebreak
\newappendix{Lampiran 2.}
\begin{spacing}{1.5}
\lipsum[1-3]
\end{spacing}
\pagebreak
\newappendix{Lampiran 3.}
\begin{spacing}{1.5}
\lipsum[1-3]
\end{spacing}
%=====================================================================
% Halaman Biodata
%\newpage
%\chapter*{BIODATA}
%\vspace{1.5pc}
\vspace{1.5pc}
\begin{spacing}{1.5}
\thispagestyle{empty}
\begin{flushleft}
\begin{tabular}{lp{0.25cm}p{9cm}}
Nama &:& Harish Abdillah Mardi\\
Tempat, tanggal lahir &:& Banda Aceh, 03 April 1999\\
Alamat &:& Komplek Dosen Desa Blangkrueng No.86, Baitussalam, Kabupaten Aceh Besar\\
Nama Ayah &:& Marwan\\
Pekerjaan Ayah &:& Dosen\\
Nama Ibu &:& Dian Mawarni\\
Pekerjaan Ibu &:& Ibu rumah Tangga\\	
Alamat Orang Tua &:& Komplek Dosen Desa Blangkrueng No.86, Baitussalam, Kabupaten Aceh Besar\\
\end{tabular}
\end{flushleft}

\noindent\hspace{0.1cm} Riwayat pendidikan :

\begin{table}[ht]
\centering
\vspace{0.1cm}
\begin{tabular}{|p{1.5cm}|p{5.3cm}|p{3cm}|p{1.5cm}|c|c|c|c|c|c|c|c|c|c|c|c|c|c|c|c|c|}
    \hline
   \centering \multirow{2}{*}{Jenjang} & \centering \multirow{2}{*}{Nama Sekolah} & \centering \multirow{2}{*}{Bidang Studi} & \multirow{2}{*}{Tempat} & Tahun\\
    &&&&Ijazah\\
    \hline
     \centering   SD&SD Swasta Sutomo 1 Medan&\centering -&\centering Medan&2011\\
    \hline
      \centering  SMP&SMP Swasta Sutomo 1 Medan&\centering -&\centering Medan&2014\\
    \hline
       \centering\multirow{2}{*}{SMA} &SMA Swasta Husni Thamrin Medan &\centering\multirow{2}{*}{IPA} &\centering\multirow{2}{*}{Medan}&\multirow{2}{*}{2017}\\
    \hline
        \centering Sarjana (S1)&Jurusan Fisika, Fakultas MIPA, Universitas Syiah Kuala & \centering Fisika Teori dan Komputasi&\centering Banda Aceh&\multirow{2}{*}{2021}\\
    \hline
\end{tabular}
\end{table}

\noindent\hspace{0.1cm} Karya tulis yang pernah dihasilkan :

\begin{table}[ht]
\centering
\vspace{0.1cm}
\begin{tabular}{|p{0.5cm}|p{7.7cm}|p{1cm}|c|}
    \hline
         \centering   No & \centering Judul & \centering Tahun & Penerbit\\
    \hline
   \centering \multirow{4}{*}{1} & Pemodelan Dispersi Gelombang pada Medium & \centering \multirow{4}{*}{2021} &\\
    &  Fiber Optik Berdasarkan Persamaan Nonlinier&& FMIPA,\\
    &   Schr\"odinger (NLS) && UNSIYIAH\\
    & (Tugas Akhir/Skripsi) &&\\
    \hline
\end{tabular}
\end{table}
\end{spacing}
%=====================================================================
\end{document}