\vspace{1.5pc}
\vspace{1.5pc}
\begin{center}
\begin{tabular}{lp{0.75\textwidth}}
  $A$ & Amplitudo \\
  $A,A_x,A_y,A_z$ & Vektor A dan vektor satuannya \\
  $a,b,c,d$ & Konstanta \\
  $A_{eff}$ & Mode area efektif \\
  $\alpha$ & Koefisien atenuasi pada fiber optik\\
  $\mathbf{B}$ & Medan magnet \\
  $B,B_x$ & Vektor B dan vektor satuannya\\
  $\beta$ & Parameter dispersi gelombang grup pada persamaan nonlinier Schr\"odinger \\
  $\beta_2$ & dispersi gelombang grup pada fiber optik\\
  $B_j$ & Kekuatan osilator\\
  $c$ & Kecepatan cahaya\\
    $c.c$ & komplex sekawan (\textit{complex conjugate})\\
  $C,C_x,C_y$ & Vektor C dan vektor satuannya\\
  $\mathbf{D}$ & Pergeseran medan listrik\\
  $\Delta$ & Operator yang menyatakan perubahan\\
  $\partial$ & Turunan parsial\\
  $\mathbf{E}$ & Medan listrik\\
  $\hat{E}$ & Invers transformasi Fourier pada medan listrik\\
  $\langle E\rangle$ & Operator energi\\
  $\epsilon_0$ & Konstanta permitivitas\\
  $\epsilon_{NL}$ & Konstanta dielektrik nonlinier\\
  $F(t)$ & Transformasi Fourier\\
  $\hat{F}(\omega)$ & Invers transformasi Fourier\\
  $H$ & Operator Hamiltonian\\
    $\mathbf{H}$ & Intensitas medan magnet\\
      $h$ & Konstanta Planck\\
  $I$ & Matriks identitas\\
    $\bar{I}(t)$ & Intensitas gelombang elektromagnetik yang berubah terhadap waktu\\
  $Im$ & Suku imajiner \\
  $i$ & Bilangan imajiner \\
  $\mathcal{J}$ & Matriks Jacobian \\
  $\mathbf{J}$ & Rapat arus pada persamaan Maxwell \\
  $\tilde{K}$ & Operator energi kinetik \\
  $k$ & Konstanta propagasi pada ansatz persamaan nonlinier Schr\"odinger \\
  $k_0$ & Konstanta propagasi pada gelombang elektromagnetik \\
  $L$ & Panjang kabel fiber optik\\
  $l$ & Jarak tempuh persatuan waktu\\
    $ln$ & Logaritma natural \\
  $\lambda$ & Panjang gelombang\\

\end{tabular}
\end{center}

\begin{center}
\begin{tabular}{lp{0.75\textwidth}}
  $\lambda\tilde{X}$ & Nilai eigen yang bekerja pada vektor eigen\\
  $M$ & Polarisasi induksi medan magnet\\
  $\mu_0$ & Konstanta permeabilitas\\
  $m$ & Massa klasik soliton\\
  $n$ & Indeks bias refraktif\\
  $\bar{n}$ & Indeks bias inti fiber optik\\
  $\tilde{n}$ & Efek indeks refraktif\\
  $\nabla$ & Operator diferensial berbasis vektor\\
  $n_g$ & Indeks bias group\\
  $\omega$ & Frekuensi sudut tak berdimensi\\
  $\omega'$ & Frekuensi sudut tak berdimensi \\
  $\omega_j$ & Frekuensi resonansi\\
  $P$ & Polarisasi induksi medan listrik\\
  $\mathbf{P}$ & Daya\\
  $\hat{P}$ & Inversi transformasi Fourier untuk polarisasi\\
  $p$ & Momentum\\
  $\mathbf{P}_{in}$ & Besar daya yang masuk\\
  $\mathbf{P}_{out}$ & Besar daya yang keluar\\

    $P_L$ & Polarisasi linier\\
  $P_{NL}$ & Polarisasi nonlinier\\
  $\Psi$ & Fungsi gelombang pada persamaan Schr\"odinger\\
  $\psi(t)$ & Amplitudo persamaan nonlinier Schr\"odinger yang mengandung variabel waktu\\
  $\phi(t)$ & Transformasi persamaan nonlinier Schr\"odinger ke dalam bentuk persamaan diferensial biasa orde 1 (pemisalan dalam perumusan sistem persamaan diferensial)\\
  $\rho$ & Massa jenis (Densitas) medium\\
  $\sigma$ & Parameter suseptibilitas bahan pada persamaan nonlinier Schr\"odinger \\
  $T$ & Waktu tunda (\textit{delay time}) \\
  $t$ & Variabel waktu tak berdimensi \\
  $t'$ & Variabel waktu tak berdimensi \\
  $\langle U\rangle$ & Kerapatan energi \\
  $\tilde{V}$ & Operator potensial \\
  $V$ & Volume medium \\
  $\nu_g$ & Gelombang grup \\
  $x$ & Variabel spasial berdimensi-$x$\\
  $\dot{x}$ & Turunan pertama fungsi $x$ \\
  $\chi$ & Parameter suseptibilitas bahan\\
  $\chi^{(1)}$ & Parameter suseptibilitas bahan orde-1\\
  $\chi^{(2)}$ & Parameter suseptibilitas bahan orde-2\\
  $\chi^{(3)}$ & Parameter suseptibilitas bahan orde-3\\
  $\chi^{(n)}$ & Parameter suseptibilitas bahan orde-n\\
  $y$ & Variabel spasial berdimensi-$y$\\
  $\dot{y}$ & Turunan pertama fungsi $y$\\
  \end{tabular}
\end{center}

\begin{center}
\begin{tabular}{lp{0.75\textwidth}}
  $\hat{y}$ & vektor polarisasi\\
  $z$ & Variabel spasial tak berdimensi\\
\end{tabular}
\end{center}